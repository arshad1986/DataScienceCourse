Which of these is a reasonable definition of machine learning?
Machine learning is the field of study that gives computers the ability to learn without being explicitly programmed.
 This was the definition given by Arthur Samuel (who had written the famous checkers playing, learning program).

A computer program is said to learn from experience E with respect to some task T and some performance measure P if its performance on T, as measured by P, improves with experience E. Suppose we feed a learning algorithm a lot of historical weather data, and have it learn to predict weather. 

The components are as follows: 

The weather prediction task. (T)
The process of the algorithm examining a large amount of historical weather data. (E)
The probability of it correctly predicting a future date's weather (P).


Classification and Regression

Question 1a
The amount of rain that falls in a day is usually measured in either millimeters (mm) or inches. Suppose you use a learning algorithm to predict how much rain will fall tomorrow. Would you treat this as a classification or a regression problem?

Classification is appropriate when we are trying to predict one of a small number of discrete-valued outputs; but in this problem, the amount of rainfall is a continuous-valued output.


Question 1b
Suppose you are working on stock market prediction, and you would like to predict the price of a particular stock tomorrow (measured in dollars). You want to use a learning algorithm for this. Would you treat this as a classification or a regression problem?


Regression is appropriate when we are trying to predict a continuous-valued output, since as the price of a stock (similar to the housing prices example in the lectures).


Supervised and Unsupervised Learning

Given 50 articles written by male authors, and 50 articles written by female authors, learn to predict the gender of a new manuscript's author (when the identity of this author is unknown).

This can be addressed as a supervised learning, classification, problem, where we learn from the labeled data to predict gender.

Given genetic (DNA) data from a person, predict the odds of him/her developing diabetes over the next 10 years.	
This can be addressed as a supervised learning, classification, problem, where we can learn from a labeled dataset comprising different people's genetic data, and labels telling us if they had developed diabetes.


Given historical data of childrens' ages and heights, predict children's height as a function of their age. 

This is a supervised learning, regression problem, where we can learn from a training set to predict height.


Given data on how 1000 medical patients respond to an experimental drug (such as effectiveness of the treatment, side effects, etc.), discover whether there are different categories or "types" of patients in terms of how they respond to the drug, and if so what these categories are.	

This can be addressed using an unsupervised learning, clustering, algorithm, in which we group the 1000 patients into different clusters based on their responses to the drug.

Take a collection of 1000 essays written on the US Economy, and find a way to automatically group these essays into a small number of groups of essays that are somehow "similar" or "related".		

This is an unsupervised learning/clustering problem (similar to the Google News example in the lectures).

Given data on how 1000 medical patients respond to an experimental drug (such as effectiveness of the treatment, side effects, etc.), discover whether there are different categories or "types" of patients in terms of how they respond to the drug, and if so what these categories are.	

This can be addressed using an unsupervised learning, clustering, algorithm, in which we group the 1000 patients into different clusters based on their responses to the drug.


Examine the statistics of two football teams, and predicting which team will win tomorrow's match (given historical data of teams' wins/losses to learn from).

This can be addressed using supervised learning, in which we learn from historical records to make win/loss predictions.

Given a large dataset of medical records from patients suffering from heart disease, try to learn whether there might be different clusters of such patients for which we might tailor separate treatements.	

This can be addressed using an unsupervised learning, clustering, algorithm, in which we group patients into different clusters.



