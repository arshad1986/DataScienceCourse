
Generalizing with multiclass classification
In this recipe, we'll look at multiclass classification. Depending on your choice of algorithm,
you either get multiclass classification for free, or you have to define a scheme for comparison.
\subsubsection*{Getting Ready}
When working with linear models such as logistic regression, we need to use
OneVsRestClassifier. This scheme will create a classifier for each class.
\subsubsection{Implementation}
First, we'll walk through a cursory example of a Decision Tree fitting a multiclass dataset.
Like we discussed earlier, we get multiclass for free with some classifiers, so we'll just fit
the example to prove that it works, and move on.
Second, we'll actually incorporate OneVsRestClassifier into our model:
>>> from sklearn import datasets
>>> X, y = datasets.make_classification(n_samples=10000, n_classes=3,
n_informative=3)
>>> from sklearn.tree import DecisionTreeClassifier
>>> dt = DecisionTreeClassifier()
>>> dt.fit(X, y)
>>> dt.predict(X)
array([1, 1, 0, .., 2, 1, 1])
As you can see, we were able to fit a classifier with minimum e\item ort.
%===================================================%
% % Classifying Data with scikit-learn
% % 146
Now, let's move on to the case of the multiclass classifier. This will require us to import
OneVsRestClassifier. We'll also import LogisticRegression while we're at it:
>>> from sklearn.multiclass import OneVsRestClassifier
>>> from sklearn.linear_model import LogisticRegression
Now, we'll override the LogisticRegression classifier. Also, notice that we can parallelize
this. If we think about how OneVsRestClassifier works, it's just training separate models
and then comparing them. So, we can train the data separately at the same time:
>>> mlr = OneVsRestClassifier(LogisticRegression(), n_jobs=2)
>>> mlr.fit(X, y)
>>> mlr.predict(X)
array([1, 1, 0, ..., 2, 1, 1])
How it works…
If we want to quickly create our own OneVsRestClassifier, how might we do it?
First, we need to construct a way to iterate through the classes and train a classifier for each
classifier. Then, we need to predict each class first:
>>> import numpy as np
>>> def train_one_vs_rest(y, class_label):
y_train = (y == class_label).astype(int)
return y_train
>>> classifiers = []
>>> for class_i in sorted(np.unique(y)):
l = LogisticRegression()
y_train = train_one_vs_rest(y, class_i)
l.fit(X, y_train)
classifiers.append(l)
Ok, so now that we have a one versus rest scheme set up, all we need to do is evaluate the
data point's likelihood for each classifier. We will then assign the classifier to the data point
with the largest likelihood.
For example, let's predict X[0]:
for classifier in classifiers
>>> print classifier.predict_proba(X[0])
[[ 0.90443776 0.09556224]]
[[ 0.03701073 0.96298927]]
[[ 0.98492829 0.01507171]]
%===================================================%
% % Chapter 4
% % 147
As you can see, the second classifier (the one in index 1) has the highest likelihood of being
"positive", therefore we'll assign 1 to this point.
