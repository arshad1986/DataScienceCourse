\documentclass[a4paper,12pt]{article}
%%%%%%%%%%%%%%%%%%%%%%%%%%%%%%%%%%%%%%%%%%%%%%%%%%%%%%%%%%%%%%%%%%%%%%%%%%%%%%%%%%%%%%%%%%%%%%%%%%%%%%%%%%%%%%%%%%%%%%%%%%%%%%%%%%%%%%%%%%%%%%%%%%%%%%%%%%%%%%%%%%%%%%%%%%%%%%%%%%%%%%%%%%%%%%%%%%%%%%%%%%%%%%%%%%%%%%%%%%%%%%%%%%%%%%%%%%%%%%%%%%%%%%%%%%%%
\usepackage{eurosym}
\usepackage{vmargin}
\usepackage{amsmath}
\usepackage{graphics}
\usepackage{epsfig}
\usepackage{framed}
\usepackage{subfigure}
\usepackage{fancyhdr}

\setcounter{MaxMatrixCols}{10}
%TCIDATA{OutputFilter=LATEX.DLL}
%TCIDATA{Version=5.00.0.2570}
%TCIDATA{<META NAME="SaveForMode"CONTENT="1">}
%TCIDATA{LastRevised=Wednesday, February 23, 201113:24:34}
%TCIDATA{<META NAME="GraphicsSave" CONTENT="32">}
%TCIDATA{Language=American English}

\pagestyle{fancy}
\setmarginsrb{20mm}{0mm}{20mm}{25mm}{12mm}{11mm}{0mm}{11mm}
\lhead{Weka and Rattle} \rhead{Kevin O'Brien} \chead{ULCIS} %\input{tcilatex}

%http://www.electronics.dit.ie/staff/ysemenova/Opto2/CO_IntroLab.pdf
\begin{document}

\tableofcontents

\section{Weka 3: Data Mining Software in Java}

Weka is a collection of machine learning algorithms for data mining tasks. The algorithms can either be applied directly to a dataset or called from your own Java code. Weka contains tools for data pre-processing, classification, regression, clustering, association rules, and visualization. It is also well-suited for developing new machine learning schemes. 

Found only on the islands of New Zealand, the Weka is a flightless bird with an inquisitive nature. The name is pronounced like this, and the bird sounds like this. 

\subsection{Licensing}
Weka is open source software issued under the GNU General Public License.

\end{document}
