
%----------------------------------------------------------%
% Page 144

Training and Testing

The error rate is just the proportions of errors made over a whole set of instances, and
it measures the overall performance of the classifiers.

%----------------------------------------------------------%
% Page 145

The error rate on the training data is called the resubstitution error, because it is calculated
by resubstituting the training instances into a classifier that was constructed from them.

%----------------------------------------------------------%
% Page 149
Cross Validation

%-------------% 
% Stratified Holdout


%----------------------------------------------------------%
% Page 152

The bootstrap

\[ \left( 1 - \frac{1}{n}\right)^n \approx e^{-1} = 0.368\]

%----------------------------------------------------------%
% Page 153

Comparing data mining methods

%----------------------------------------------------------%
% Page 157

The corrected resampled $t-$test uses the modified statistic

\[ t = \frac{\bar{d}}{\sqrt{ \left(\frac{1}{k} + \frac{n_2}{n_1}\right)\sigma^2_d}}

%------------% Page 157 Bottom
Predicting Probabilities

%----------------------------------------------------------%
% Page 158
Quadratic Loss Function


%----------------------------------------------------------%
% Page 159
Informational Loss Function

Another popular criterion for the evaluation of probabilistic prediction is the \textit{informational loss function}.

\[ -log_2P_i\]

where the $i$th prediction is the correct one

%----------------------------------------------------------%
% Page 161
Minimum Description Length (MDL)



%-------------------%

Counting the Cost



%----------------------------------------------------------%
% Page 164
Cost-sensitive classification

%----------------------------------------------------------%
% Page 165
Cost-sensitive learning

%----------------------------------------------------------%
% Page 166 
Lift Charts

% Page 168
ROC Curves

Lift charts are a valuable tool, widely used in marketing.

The receiving operating characteristic curve.

ROC curves depict the performance of a classifier without regard to class distribution or error costs.

%Page 177

mean squared error

mean absolute error
relative squared error


%---------------------------------------------------------%





%---------------------------------------------------------%
%Page 183
Applying the MDL principle to Clustering





