Big Data refers to a collection of data sets so large and complex, it’s impossible to process them with the usual databases 
and tools. Because of its size and associated numbers, Big Data is hard to capture, store, search, share, analyze and 
visualize. 

The phenomenon came about in recent years due to the sheer amount of machine data being generated today – thanks
to mobile devices, tracking systems, RFID, sensor networks, social networks, Internet searches, automated record keeping,
video archives, e-commerce, etc. – coupled with the additional information derived by analyzing all this information, 
which on its own creates another enormous data set. 

Companies pursue Big Data because it can be revelatory in spotting business trends, improving research quality, 
and gaining insights in a variety of fields, from IT to medicine to law enforcement and everything in between and beyond.
