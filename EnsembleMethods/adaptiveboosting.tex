 \section{AdaBoost}
 
\textbf{\textit{AdaBoost}}, short for "Adaptive Boosting", is a machine learning meta-algorithm formulated by Yoav Freund and Robert Schapire who won the prestigious "Gödel Prize" in 2003 for their work. 

It can be used in conjunction with many other types of learning algorithms to improve their performance. The output of the other learning algorithms ('weak learners') is combined into a weighted sum that represents the final output of the boosted classifier. 

\textbf{\textit{AdaBoost}} is adaptive in the sense that subsequent weak learners are tweaked in favor of those instances misclassified by previous classifiers. AdaBoost is sensitive to noisy data and outliers. In some problems, however, it can be less susceptible to the overfitting problem than other learning algorithms. 

The individual learners can be weak, but as long as the performance of each one is slightly better than random guessing (i.e., their error rate is smaller than 0.5 for binary classification), the final model can be proven to converge to a strong learner.

While every learning algorithm will tend to suit some problem types better than others, and will typically have many different parameters and configurations to be adjusted before achieving optimal performance on a dataset, AdaBoost (with decision trees as the weak learners) is often referred to as the best out-of-the-box classifier. 

When used with decision tree learning, information gathered at each stage of the AdaBoost algorithm about the relative 'hardness' of each training sample is fed into the tree growing algorithm such that later trees tend to focus on harder to classify examples.

%===================================%
