\subsection{Main Concepts of Torch}
%http://citeseerx.ist.psu.edu/viewdoc/download;jsessionid=CBB0C8A5FE34F6D6DAFF997F6B6A205A?doi=10.1.1.8.9850&rep=rep1&type=pdf

Torch has been developed using an object-oriented paradigm and implemented in C++. In order to
simplify the modification of existing algorithms or the design of new algorithms or methods, a modular
strategy was chosen through the denition of the following broad classes:

\begin{descrption}
\item[DataSet:] this 
lass handles data. Several sub
lasses provide ways to handle stati
 or dynami

data, data that 
an t into memory or whi
h 
ould be a

essed on-the-y from disk (for very
large data sets for example), et
...
 Ma
hine: this 
lass represents any bla
k-box that, given an (optional) input (again, either stati

or dynami
) and some (optional) parameters, returns an output. It 
ould be for instan
e a
neural network, a support ve
tor ma
hine, a hidden Markov model, et
...
 Trainer: this 
lass is used to sele
t an optimal set of parameters of a ma
hine a

ording to a
given 
riterion and a given DataSet, and test it using another (or the same) DataSet.
 Measurer: ob je
ts of this 
lass print in dierent les various measures of interest. It 
ould be
for example the 
lassi
ation error, the mean-squared error or the log-likelihood.
Thus, the general idea of Tor
h is very simple: rst the DataSet produ
es one or several training
examples. The Trainer gives them to the Ma
hine whi
h 
omputes an output, whi
h is used by the
Trainer to tune the parameters of the Ma
hine. During this pro
ess, one or more Measurer(s) 
an
be used to monitor the performan
e of the system. Note however that some ma
hines 
an only be
trained by spe
i
 trainers:
