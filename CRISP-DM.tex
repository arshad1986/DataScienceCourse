%---------------------------%
\section{CRISP-DM}

CRISP-DM stands for CRoss Industry Standard Process for Data Mining. It is a data mining process model that describes commonly used approaches that expert data miners use to tackle problems. Polls conducted in 2002, 2004, and 2007 show that it is the leading methodology used by data miners

CRISP-DM breaks the process of data mining into six major phases:

\begin{enumerate}
\item Business Understanding
\item Data Understanding
\item Data Preparation
\item Modeling
\item Evaluation
\item Deployment
\end{enumerate}

\subsection{Business Understanding}

This initial phase focuses on understanding the project objectives and requirements from a business perspective, and then converting this knowledge into a data mining problem definition, and a preliminary plan designed to achieve the objectives.


\subsection{Data Understanding}

The data understanding phase starts with an initial data collection and proceeds with activities in order to get familiar with the data, to identify data quality problems, to discover first insights into the data, or to detect interesting subsets to form hypotheses for hidden information.


\subsection{Data Preparation}

The data preparation phase covers all activities to construct the final dataset (data that will be fed into the modeling tool(s)) from the initial raw data. Data preparation tasks are likely to be performed multiple times, and not in any prescribed order. Tasks include table, record, and attribute selection as well as transformation and cleaning of data for modeling tools.


\subsection{Modeling}

In this phase, various modeling techniques are selected and applied, and their parameters are calibrated to optimal values. Typically, there are several techniques for the same data mining problem type. Some techniques have specific requirements on the form of data. Therefore, stepping back to the data preparation phase is often needed.

\subsection{Evaluation}

At this stage in the project you have built a model (or models) that appears to have high quality, from a data analysis perspective. Before proceeding to final deployment of the model, it is important to more thoroughly evaluate the model, and review the steps executed to construct the model, to be certain it properly achieves the business objectives. A key objective is to determine if there is some important business issue that has not been sufficiently considered. At the end of this phase, a decision on the use of the data mining results should be reached.

\subsection{Deployment}

Creation of the model is generally not the end of the project. Even if the purpose of the model is to increase knowledge of the data, the knowledge gained will need to be organized and presented in a way that the customer can use it. Depending on the requirements, the deployment phase can be as simple as generating a report or as complex as implementing a repeatable data mining process. In many cases it will be the customer, not the data analyst, who will carry out the deployment steps. However, even if the analyst will not carry out the deployment effort it is important for the customer to understand up front what actions will need to be carried out in order to actually make use of the created models.



\subsection{Data Mining Concepts}

The process is known as Knowledge Discovery in Databases and was developed by Gregory Piatetsky-Shapiro in 1989. Four different classes of data mining concepts allow the process to take place. Clustering uses the algorithm created from the data mining process to assemble items into similar groups.
