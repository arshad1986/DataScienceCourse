\documentclass[12pt]{article}

%opening
\title{Data Science}
\author{Kevin O'Brien}

\begin{document}
	\large
\subsection*{Data Mapping}
Data mapping is the process by which two distinct data models are created and a link between these models is defined. Data models can include either metadata, an atomic unit of data with a precise meaning in regards to semantics, and telecommunications. The system uses the atomic unit system to measure the properties of electricity which contain the information. 


Data mapping is most readily used in software engineering to describe the best way to access or represent some form of information. It works as an abstract model to determine relationships within a certain domain of interest. This is the fundamental first step in establishing data integration of a particular domain.


The main uses for data mapping include a wide variety of platforms. Data transformation is used to mediate the relationship between an initial data source and the destination in which that data is used. It is useful in identifying parts of data lineage analysis, the way in which data flows from one sector of information to another. Data mapping is also integral in discovering hidden information and sensitive data such as social security numbers when hidden within a different identification format. This is known as data masking.


Certain procedures are put in place when data mapping is conducted. This allows a user to create or transform the information into a form in which the best results can be culled. Commonly, this takes the form of some graphical mapping tool that is able to automatically generate results and execute a transformation of the data. Essentially, a user is able to literally “draw” a line from one field to another, identifying the correct connection. This is known as manual data mapping.

In regards to the basic mapping technique of a data element, a number of specific formula considerations need to be addressed. The data element itself needs to identified and named, a clear definition of the data needs to be determined and representation of the values are enumerated. In some terms, the identifiers are represented in the form of a database. Standard structures are built with basic units of information, such as names, addresses or ages.
\subsubsection{Example}
For example, when a company merges with another company, they need to merge data for both sets of customers. Data mapping can be used to track one set of information and cross-reference it with another set of data. This allows both companies to merge the data into one final database.

One of the newest techniques in data mapping involves using statistics simultaneously with two values of divergent data sources. This allows more complex mapping operations between the two data sets. It can be highly valued when it comes to discovering more specialized informational aspects such as substrings.
%--------------------------------------------------------%
\end{document}