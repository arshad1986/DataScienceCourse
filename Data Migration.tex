
\documentclass[12pt]{article}

%opening
\title{Data Science}
\author{Kevin O'Brien}

\begin{document}
	
	\section*{Data Migration}

\begin{itemize}
\item Data migration means what it sounds like it means -- sort of. It's not data that moves one from place to another, unless you think of places as being virtual. Data migration is actually the translation of data from one format to another format or from one storage device to another storage device. This also necessarily requires someone or something to do the translating. Data doesn't just get up and walk to another format all by itself.



\item Data migration is necessary when a company upgrades its database or system software, either from one version to another or from one program to an entirely different program. Software can be specifically written, using entire programs or just scripts, to facilitate data migration. Such upgrades or program switches can take place as a result of regular company practices or as a result of directives mandated in the wake of a company takeover.



\item Another use of data migration is to store little-used data on magnetic tape or other backup storage methods. This data may need to be stored for historical purposes or for periodic access. Individual computer users do this all the time when they back up their data to CDs, DVDs, or external hard drives. Companies large and small do this, of course, to protect and archive their data. Migrated data typically is moved offline but remains available via network access, leaving the online environment free to conduct current business.



\item Data migration typically has four phases: analysis of source data, extraction and transformation of data, validation and repair of data, and use of data in the new program. During each phase, the data migration software works its electronic magic, performing the necessary machinations before moving the data on through the process. Perhaps the most sensitive phase is validation and repair. In this phase, data is evaluated for potential problems, which are flagged and identified to the user. Unresolvable problems can be identified at this stage, along with untranslatable data, so they can be set aside and not gum up the whole data migration works.
\end{itemize}

\end{document}