Model Fit Diagnostics
======================

- Nagelkerke R-Squared


One question that obviously arises when we are looking at regression models is that of the overall fit of the model. As in ordinary linear regression, we can find measures of model fit of logistic regression models.

There are, in fact, a number of different measures of goodness of fit for logistic regression models. The first, most straightforward measure, is to simply look at the extent to which the model accurately predicts the dependent variable, or, in other words, how accurate the model is at predicting whether or not a pupil, in this sample, is likely to report having literature in their home. This is calculated by comparing the predicted score of the individual students (as either possessing or not possessing literature) on the basis of the two independent variables we have in our model (gender and mother’s education), with their actual group membership as given by the data (in other words, what the data tells us about whether students have actually said they possess or don’t possess literature in the home). 

Hosmer-Lemeshow Test
 One of these measures is the Hosmer-Lemeshow test of goodness of fit. This is similar to a Chi Square test, and indicates the extent to which the model provides better fit than a null model with no predictors, or, in a different interpretation, how well the model fits the data, as in log-linear modelling. If chi-square goodness of fit is not significant, then the model has adequate fit. By the same token, if the test is significant, the model does not adequately fit the data. The Hosmer and Lemeshow's (H-L) goodness of fit test divides subjects into deciles based on predicted probabilities, then computes a chi-square from observed and expected frequencies. Then a probability (p) value is computed from the chi-square distribution to test the fit of the logistic model. If the H-L goodness-of-fit test statistic is greater than .05, as we want for well-fitting models, we fail to reject the null hypothesis that there is no difference between observed and model-predicted values, implying that the model's estimates fit the data at an acceptable level. That is, well-fitting models show nonsignificance on the goodness-of-fit test, indicating model prediction that is not significantly different from observed values.

A disadvantage of this goodness of fit measure is that it is a significance test, with all the limitations this entails. Like other significant tests it only tells us whether the model fits or not, and does not tell us anything about the extent of the fit. Similarly, like other significance tests, it is strongly influenced by the sample size (sample size and effect size both determine significance), and in large samples, such as the PISA dataset we are using here, a very small difference will lead to significance. As the sample size gets large, the H-L statistic can find smaller and smaller differences between observed and model-predicted values to be significant. Small sample sizes are also problematic, however, as, being a Chi Square test we can’t have too many groups (more than 10%) with predicted frequencies of less than five.

The Hosmer-Lemeshow test of goodness of fit is not automatically a part of the SPSS logistic regression output. To get this output, we need to go into ‘options’ and tick the box marked Hosmer-Lemeshow test of goodness of fit. In our example, this gives us the following output:

 Step	 Chi-square	df 	 Sig.
 1	 142.032	 6	 .000
Therefore, our model is significant, suggesting it does not fit the data. However, as we have a sample size of over 13,000, even very small divergencies of the model from the data would be flagged up and cause significance. Therefore, with samples of this size it is hard to find models that are parsimonious (i.e. that use the minimum amount of independent variables to explain the dependent variable) and fit the data. Therefore, other fit indices might be more appropriate.

 In ordinary linear regression, our primary measure of model fit was R2, which was an indicator of the percentage of variance in the dependent variable explained by the model. It would be useful for us to have a similar measure for logistic regression. However, the R2 measure is only appropriate to linear regression, with its continuous dependent variables. To get around this problem, a number of statisticians have developed so-called ‘Pseudo R2 ’ measures that aim to mimic R2 for logistic regression models. In contrast to the actual R2 , as these are approximations there are a number of different Pseudo R Squares, which take a different conceptual approach to what R2 means .

The most commonly used interpret R2 as representing the improvement of the model we are using (in our case the two variable model) over the null model with no independent variables (also called predictors). Other approaches are based on viewing R2 as explained variance.

In the first category we can find two Pseudo R2   measures used in SPSS, Cox and Snell’s and Nagelkerke’s Pseudo R square measures. Cox and Snell's R2 is based on calculating the proportion of unexplained variance that is reduced by adding variables to the model.

The formula for Cox and Snell's Pseudo R2 is:

where -2LLnull is the loglikelihood for the empty model, and -2LLk is the loglikelihood for the model with the independent variables.

There is a major problem with Cox and Snell’s Pseudo R Square, however, which is that its maximum can be (and usually is) less than 1.0, making it difficult to interpret. That is why Nagelkerke developed a modified version of Cox and Snell’s measure that varies from 0 to 1. To achieve this, Nagelkerke's R 2 divides Cox and Snell's R 2 by its maximum, giving us the formula:

Therefore Nagelkerke's Pseudo R2 will normally be higher than the Cox and Snell measure. Both of these Pseudo R2 measures will tend to be lower than traditional ordinary least squares R2 measures.


<hr>
### Effron's Pseudo R-squared
An example of an approach that views R2 as explained variance is Effron’s Pseudo R2. This measure takes the model residuals, which are squared, summed, and divided by the total variability in the dependent variable:


This R-squared is equal to the squared correlation between the predicted values and actual values. 

Other Pseudo R2s, such as Mc Fadden’s and McKelvey and Zavoina’s measures also exist, but we will not discuss them all in detail here. However, the existence of multiple measures, as opposed to the one R2 we had in ordinary linear regression points to the facts that these are approximations of R2, which are inexact and disputable to some extent, and it is important to remember that they will give us somewhat different numbers.

<hr>



### Pseudo R-Squared in SPSS
In SPSS we only get two Pseudo R2 measures in the output, Cox and Snell and Nagelkerke. These are given in the box labelled ‘model summary’ in the output:

Step

-2 Log likelihood

Cox & Snell R Square

Nagelkerke R Square

1

16327.952(a)

.055

.074

As we can see, Nagelkerke’s measure gives us a higher value than does Cox and Snell’s as we would expect. We also said earlier that Nagelkerke’s measure was a correction of Cox and Snell’s, allowing the measure to use the full 0-1 range, so we will prefer to use this one.

Whichever of the measures we use, however, we can see that fit of the model is poor. As .07 is close to 0 our model is not a great improvement over the null model with no predictors.

Summarising what these three measures tell us about the fit of our model, the Hosmer and Lemesnow Goodness of Fit Chi Square test indicates that our model does not fit the data. However, this measure is sensitive to sample size, and in our large sample few parsimonious models would fit. Nevertheless, poor fit is also indicated by the other measures. Accuracy of prediction has improved over the null model, but only by 4%. Nagelkerke’s Pseudo R2 is only .07, again (think of an analogous R2in ordinary linear regression) indicating poor fit. So, even though both our predictors are significant, they are weak predictors of possessing literature in the home.

### References
- http://www.statisticalhorizons.com/r2logistic
