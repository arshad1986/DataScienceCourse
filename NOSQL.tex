\section{What is NoSQL?}
\textbf{NoSQL} stands for \textbf{\textit{Not Only SQL}}, a database paradigm that expands upon the relation database design.

%-------------------------------------------------%
\section{Types of NoSQL}
\begin{itemize}
\item MongoDB
\item Cassandra
\item Riak!
\end{itemize}
\newpage
\section{Important Concepts for NoSQL}

%-------------------------------------------------%
NoSQL (Not Only SQL)
Data Storage: The world's stored digital data is measured in exabytes. An exabyte is equal to one billion gigabytes (GB) of data. According to Internet.com, the amount of stored data added in 2006 was 161 exabytes. Just 4 years later in 2010, the amount of data stored will be almost 1,000 ExaBytes which is an increase of over 500%. In other words, there is a lot of data being stored in the world and its just going to continue growing.

Interconnected Data: Data continues to become more connected. The creation of the web fostered in hyperlinks, blogs have pingbacks and every major social network system has tags that tie things together. Major systems are built to be interconnected.
Complex Data Structure: NoSQL can handle hierarchical nested data structures easily. To accomplish the same thing in SQL, you would need multiple relational tables with all kinds of keys. In addition, there is a relationship between performance and data complexity. Performance can degrade in a traditional RDBMS as we store the massive amounts of data required in social networking applications and the semantic web.








Hive HiveQL
netezza
teradata
map reduce algorithm
Hadoop Platform
Enterprise R
HBase: open source equivalent of Google's BigTable.  


\end{document}
