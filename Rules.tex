


\subsection{Association Rules}

Association rules are if/then statements that help uncover relationships between seemingly unrelated data in a relational database or other information repository. An example of an association rule would be ``If a customer buys a dozen eggs, he is $80\%$ likely to also purchase milk."

An association rule has two parts, an \textbf{\emph{antecedent (if)}} and a \textbf{\emph{consequent (then)}}. An antecedent is an item found in the data. A consequent is an item that is found in combination with the antecedent.

Association rules are created by analyzing data for frequent if/then patterns and using the criteria support and confidence to identify the most important relationships. Support is an indication of how frequently the items appear in the database. Confidence indicates the number of times the if/then statements have been found to be true.

In data mining, association rules are useful for analyzing and predicting customer behavior. They play an important part in shopping basket data analysis, product clustering, catalog design and store layout.

Programmers use association rules to build programs capable of machine learning.  Machine learning is a type of artificial intelligence (AI) that seeks to build programs with the ability to become more efficient without being explicitly programmed.
