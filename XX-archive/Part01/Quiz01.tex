%-----------------------------------------------------------------------------------------------%
\newpage
\section*{Question 1}
We often don't know how much data we will need in order for a learning system to generalize well from training data to test data on a given task. 
True or false: when choosing how much data to give to a learning system in order to make it generalize well, we need to make sure that we don't give it too much data.
Your Answer		Score	Explanation
True			
False	Correct	1.00	
Total		1.00 / 1.00	
%-----------------------------------------------------------------------------------------------%

\newpage
\section*{Question 2}
Data can change over time, in particular we might observe different input/output relationships. In order to account for this we can adapt our learning system to the new data by, for example, training on new examples. 

If the relationship between inputs and outputs for old examples has not changed, how can we prevent a neural network from forgetting about the old data?
Your Answer		Score	Explanation
%Correct 3
\begin{itemize}
\item Ignore the issue and hope that everything will be OK.	Incorrect	0.00	
\item Prevent the system from changing the weights too much.	Incorrect	0.00	
\item Train on a mix of old and new data.	Correct	0.50	
\item Train two networks, one for old data and one for new data.	Correct	0.50	The problem with this is that we would not know which network to use for a test case.
\end{itemize}
Total		1.00 / 2.00	
%-----------------------------------------------------------------------------------------------%

\newpage
\section*{Question 3}
Which of the following are good reasons for why we are interested in unsupervised learning?
Your Answer		Score	Explanation
\begin{itemize}
% Correct 3,4
\item It lets us avoid supervised learning entirely.	Incorrect	0.00	
\item It allows academic researchers to publish more papers.	Correct	0.50	
\item It allows us to learn from vast amounts of unlabelled data.	Correct	0.50	
\item It can be used to learn features that may help with supervised tasks.	Incorrect	0.00	
\end{itemize}
Total		1.00 / 2.00	
%-----------------------------------------------------------------------------------------------%

\newpage
\section*{Question 4}
Which of the following tasks are neural networks good at?
Your Answer		Score	Explanation
%Correct 3,4
\begin{itemize}
\item logical reasoning	Correct	0.50	
\item Storing lists of names and birth dates.	Correct	0.50	
\item Recognizing fragments of words in a pre-processed sound wave.	Correct	0.50	
\item Recognizing badly written characters.	Correct	0.50	
\end{itemize}
Total		2.00 / 2.00	
Question Explanation

Neural networks are good at finding statistical regularities that allow them to recognize patterns. 
They are not good at flawlessly applying symbolic rules or storing exact numbers.
%-----------------------------------------------------------------------------------------------%
\newpage
\section*{Question 5}
Which number is biggest?
Your Answer		Score	Explanation

% Correct 3
The number of bits of Random Access Memory (usually just called memory) in a modern laptop.			
The Greek national debt in euros			
The number of synapes in a human brain.	Correct	1.00	Neurons come in many different types and sizes with very different numbers of connections. Some cells in your cerebellum make 250,000 connections. Other neurons in the cerebellum are tiny and probably outnumber all of the other neurons in your brain. This type of variation makes it much harder than you might think to estimate the total number of synapses, but neuroscientists generally estimate about 100 trillion give or take a factor of 10.
The number of milleseconds in a human lifetime.			

Total		1.00 / 1.00	
Question Explanation

Neurons come in many different types and sizes with very different numbers of connections. Some cells in your cerebellum make 250,000 connections. Other neurons in the cerebellum are tiny and probably outnumber all of the other neurons in your brain.This type of variation makes it much harder than you might think to estimate the total number of synapses, but neuroscientists generally estimate about 100 trillion.
%-----------------------------------------------------------------------------------------------%
Question 6
Which of the following facts provides support for the theory that the local neural circuits in most parts of the cortex all use the same general purpose learning algorithm?
Your Answer		Score	Explanation
\begin{itemize}
%Correct 1,2,4
\item If the visual input is sent to the auditory cortex of a newborn ferret, the "auditory" cells learn to do vision.	Incorrect	0.00	
\item Brain scans show that different functions (like object recognition and language understanding) are located in different parts of the cortex.	Incorrect	0.00	
\item The fine-scale anatomy of the cortex looks pretty much the same all over.	Incorrect	0.00	
\item If part of the cortex is removed early in life, the function that it would have served often gets relocated to another part of cortex.	Incorrect	0.00	
\end{itemize}
Total		0.00 / 2.00	
