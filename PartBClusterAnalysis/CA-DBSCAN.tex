Density-based spatial clustering of applications with noise (DBSCAN) is a data clustering algorithm proposed by Martin Ester, Hans-Peter Kriegel, Jörg Sander and Xiaowei Xu in 1996.[1] It is a density-based clustering algorithm because it finds a number of clusters starting from the estimated density distribution of corresponding nodes. DBSCAN is one of the most common clustering algorithms and also most cited in scientific literature.

%======================================================================================================%

The DBSCAN algorithm views clusters as areas of high density separated by areas of low density. 

Due to this rather generic view, clusters found by DBSCAN can be any shape, as opposed to k-means which assumes that clusters are convex shaped. 
The central component to the DBSCAN is the concept of core samples, which are samples that are in areas of high density. 

A cluster is therefore a set of core samples, each close to each other (measured by some distance measure) and a set of non-core samples that are close to a core sample (but are not themselves core samples). There are two parameters to the algorithm, min_samples and eps, which define formally what we mean when we say dense. A higher min_samples or lower eps indicate higher density necessary to form a cluster.
More formally, we define a core sample as being a sample in the dataset such that there exist min_samples other samples within a distance of eps, which are defined as neighbors of the core sample. This tells us that the core sample is in a dense area of the vector space. A cluster is a set of core samples, that can be built by recursively by taking a core sample, finding all of its neighbors that are core samples, finding all of their neighbors that are core samples, and so on. A cluster also has a set of non-core samples, which are samples that are neighbors of a core sample in the cluster but are not themselves core samples. Intuitively, these samples are on the fringes of a cluster.
Any core sample is part of a cluster, by definition. 

Further, any cluster has at least min_samples points in it, following the definition of a core sample. 
For any sample that is not a core sample, and does have a distance higher than eps to any core sample, it is considered an outlier by the algorithm.

% http://scikit-learn.org/stable/modules/clustering.html
