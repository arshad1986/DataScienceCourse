
% See "book", "report", "letter" for other types of document.

\documentclass[11pt]{article} % use larger type; default would be 10pt

\usepackage[utf8]{inputenc} % set input encoding (not needed with XeLaTeX)

\usepackage{geometry} % to change the page dimensions
\geometry{a4paper} % or letterpaper (US) or a5paper or....

\usepackage{graphicx} % support the \includegraphics command and options


\usepackage{booktabs} % for much better looking tables
\usepackage{array} % for better arrays (eg matrices) in maths
\usepackage{paralist} % very flexible & customisable lists (eg. enumerate/itemize, etc.)
\usepackage{verbatim} 
\usepackage{subfig} 
\usepackage{fancyhdr} % This should be set AFTER setting up the page geometry
\pagestyle{fancy} % options: empty , plain , fancy
\renewcommand{\headrulewidth}{0pt} % customise the layout...
\lhead{KNIME}\chead{Introduction to KNIME}\rhead{Dragonfly Statistics}
\lfoot{}\cfoot{\thepage}\rfoot{}

\usepackage{sectsty}
\allsectionsfont{\sffamily\mdseries\upshape} % (See the fntguide.pdf for font help)
% (This matches ConTeXt defaults)

%%% ToC (table of contents) APPEARANCE
\usepackage[nottoc,notlof,notlot]{tocbibind} % Put the bibliography in the ToC
\usepackage[titles,subfigure]{tocloft} % Alter the style of the Table of Contents
\renewcommand{\cftsecfont}{\rmfamily\mdseries\upshape}
\renewcommand{\cftsecpagefont}{\rmfamily\mdseries\upshape} % No bold!

\title{KNIME}
\author{Dragonfly Statistics}

% CURRENTLY RE_WRITING

%-----------------------------------------------%
\begin{document}
\subsection*{KNIME Architecture} 
The architecture of Knime was designed with three main
principles in mind:
\begin{enumerate}
\item  visual, interactive framework: data flows should be combined
by simple drag\&drop from a variety of processing
units. Customized applications can be modelled through
individual data pipelines.
\item modularity: Processing units and data containers should
not depend on each other in order to enable easy distribution
of computation and allow for independent development
of different algorithms. Data Types are encapsulated,
that is, no types are predefined, new types can easily be
added bringing along type specific renderers and comparators.
New types can be declared compatible to existing
types.
\item  easy expandability: It should be easy to add new processing
nodes, or views and distribute them through a simple
plug\&play principle without the need for complicated \textit{install/
deinstall} procedures.
\end{enumerate}
\subsection*{KNIME Data Structures}
All data flowing between nodes is wrapped within a class
called \texttt{DataTable} which holds meta-information concerning
the type of its columns and the actual data. 

The data can
be accessed by iterating over instances of \texttt{DataRow}. Each
row contains a unique identifier (or \texttt{primary key}) and a
specific number of \texttt{DataCell} objects which hold the actual
data. 

The reason to avoid access by Row ID or index is
scalability, that is, the desire to be able to process large
amounts of data and therefore not be forced to keep all of
the rows in memory for fast, random access.

\end{document}
