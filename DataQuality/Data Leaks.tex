

\documentclass[12pt]{article}

%opening
\title{Data Science}
\author{Kevin O'Brien}

\begin{document}
	\section*{Data Leaks}
	
	\begin{itemize}
\item A data leak, also called a data Valdez, is a term used to describe the accidental publishing of private information online. This problem has happened several times since the advent of the internet, and is potentially dangerous to the victims. Personal information, search records and health records have all been published through data Valdez incidents. Usually, the leaks are a result of either programming errors or the work of hackers.

 

\item The term, data Valdez, refers to the infamous oil spill caused by the oil tanker the Exxon Valdez. In this incident, over ten million gallons of oil were spilled into the Prince William Sound off the coast of Alaska. Like the oil spill, a data Valdez is impossible to get back once the leak has occurred.

 

\item One of the earliest large-scale incidents of a data Valdez was in 2005, when a laptop was stolen from an Ameriprise Financial employee. The laptop was discovered to contain about 260,000 confidential customer records including account records. In August 2006, the laptop was retrieved by law enforcement, but the incident raised the public awareness level of data leaks.

 

\item Many college and university campuses have been subject to several data leak incidents, resulting in the compromising of thousands of student records. One of the first was at George Mason University in Virginia, where at least 32,000 records were compromised because of a hacker attack. Hackers and hard-drive thefts are responsible for many university data Valdez incidents, including a 2006 attack on the University of California, Los Angeles (UCLA,) in which 800,000 records were compromised that contained present and past student names, addresses, social security numbers, and financial aid information.

 

\item Many times, data Valdez incidents occur due to mistakes by company officials. In some incidents, customer information has been accidentally posted or sent as an email attachment to unintended recipients. This type of mistake is fairly easy to make, as many email programs contain a “reply to all” button, which with one click can send a private email company-wide. However, such mistakes have resulted in the firing of the responsible employee and occasional lawsuits against offending companies.

 

\item In most cases, the compromised data is not actually used for any nefarious purpose. With hackers, many times the point of their breach is simply to prove that they can subvert security systems. Many of the data Valdez incidents are also the result of random laptop theft, which will usually not result in consequences for the compromised victims. However, the exposure of personal data is something to be extremely concerned about in case of identity theft. If you discover you are the victim of a data leak, it is recommended that you run credit card checks, put a watch system on all credit cards, and report any suspicious activity done in your name to law enforcement.
\end{itemize}
\end{document}