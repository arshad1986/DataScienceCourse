Partitioning clustering are clustering methods used to classify observations, within a data set, into multiple groups based on their similarity. The algorithms require the analyst to specify the number of clusters to be generated.

This chapter describes the commonly used partitioning clustering, including:

* K-means clustering (MacQueen 1967), in which, each cluster is represented by the center or means of the data points belonging to the cluster. The K-means method is sensitive to anomalous data points and outliers.

* K-medoids clustering or PAM (Partitioning Around Medoids, Kaufman & Rousseeuw, 1990), in which, each cluster is represented by one of the objects in the cluster. PAM is less sensitive to outliers compared to k-means.

* CLARA algorithm (Clustering Large Applications), which is an extension to PAM adapted for large data sets.

For each of these methods, we provide:

* the basic idea and the key mathematical concepts
* the clustering algorithm and implementation in R software
* R lab sections with many examples for cluster analysis and visualization

The following R packages will be used to compute and visualize partitioning clustering:

stats package for computing K-means
cluster package for computing PAM and CLARA algorithms
factoextra for beautiful visualization of clusters
