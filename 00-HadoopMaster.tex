Apache Hadoop is an open-source software framework for storage and large-scale processing of data-sets on clusters of commodity hardware.
%-----------------------------------------------------%
\section{The Hadoop Ecosysmtem}
The term Hadoop has come to refer not just to the base modules above, but also to the ecosystem,[8] or collection of additional software packages that can be installed on top of or alongside Hadoop, such as Apache Pig, Apache Hive, Apache HBase, Apache Phoenix, Apache Spark, Apache ZooKeeper, Cloudera Impala, Apache Flume, Apache Sqoop, Apache Oozie, Apache Storm.[9]

\subsection{HDFS}

HDFS is the primary distributed storage used by Hadoop applications. A HDFS cluster primarily consists of a NameNode that manages the file system metadata and DataNodes that store the actual data. The HDFS Architecture Guide describes HDFS in detail. This user guide primarily deals with the interaction of users and administrators with HDFS clusters. The HDFS architecture diagram depicts basic interactions among NameNode, the DataNodes, and the clients. Clients contact NameNode for file metadata or file modifications and perform actual file I/O directly with the DataNodes.

The following are some of the salient features that could be of interest to many users.

Hadoop, including HDFS, is well suited for distributed storage and distributed processing using commodity hardware. It is fault tolerant, scalable, and extremely simple to expand. MapReduce, well known for its simplicity and applicability for large set of distributed applications, is an integral part of Hadoop.
HDFS is highly configurable with a default configuration well suited for many installations. Most of the time, configuration needs to be tuned only for very large clusters.
Hadoop is written in Java and is supported on all major platforms.
Hadoop supports shell-like commands to interact with HDFS directly.
The NameNode and Datanodes have built in web servers that makes it easy to check current status of the cluster.
New features and improvements are regularly implemented inHDFS. The following is a subset of useful features inHDFS:
File permissions and authentication.
Rack awareness: to take a node’s physical location into account while scheduling tasks and allocating storage.
Safemode: an administrative mode for maintenance.
fsck: a utility to diagnose health of the file system, to find missing files or blocks.
fetchdt: a utility to fetch DelegationToken and store it in a file on the local system.
Rebalancer: tool to balance the cluster when the data is unevenly distributed among DataNodes.
Upgrade and rollback: after a software upgrade, it is possible to rollback to HDFS’ state before the upgrade in case of unexpected problems.
Secondary NameNode: performs periodic checkpoints of the namespace and helps keep the size of file containing log of HDFS modifications within certain limits at the NameNode.
Checkpoint node: performs periodic checkpoints of the namespace and helps minimize the size of the log stored at the NameNode containing changes to the HDFS. Replaces the role previously filled by the Secondary NameNode, though is not yet battle hardened. The NameNode allows multiple Checkpoint nodes simultaneously, as long as there are no Backup nodes registered with the system.
Backup node: An extension to the Checkpoint node. In addition to checkpointing it also receives a stream of edits from the NameNode and maintains its own in-memory copy of the namespace, which is always in sync with the active NameNode namespace state. Only one Backup node may be registered with the NameNode at once.

%=====================================================%


 Previous Page Next Page  
Hadoop File System was developed using distributed file system design. It is run on commodity hardware. Unlike other distributed systems, HDFS is highly faulttolerant and designed using low-cost hardware.

HDFS holds very large amount of data and provides easier access. To store such huge data, the files are stored across multiple machines. These files are stored in redundant fashion to rescue the system from possible data losses in case of failure. HDFS also makes applications available to parallel processing.

Features of HDFS
It is suitable for the distributed storage and processing.
Hadoop provides a command interface to interact with HDFS.
The built-in servers of namenode and datanode help users to easily check the status of cluster.
Streaming access to file system data.
HDFS provides file permissions and authentication.
HDFS Architecture
Given below is the architecture of a Hadoop File System.

HDFS Architecture
HDFS follows the master-slave architecture and it has the following elements.

Namenode
The namenode is the commodity hardware that contains the GNU/Linux operating system and the namenode software. It is a software that can be run on commodity hardware. The system having the namenode acts as the master server and it does the following tasks:

Manages the file system namespace.
Regulates client’s access to files.
It also executes file system operations such as renaming, closing, and opening files and directories.
Datanode
The datanode is a commodity hardware having the GNU/Linux operating system and datanode software. For every node (Commodity hardware/System) in a cluster, there will be a datanode. These nodes manage the data storage of their system.

Datanodes perform read-write operations on the file systems, as per client request.
They also perform operations such as block creation, deletion, and replication according to the instructions of the namenode.
Block
Generally the user data is stored in the files of HDFS. The file in a file system will be divided into one or more segments and/or stored in individual data nodes. These file segments are called as blocks. In other words, the minimum amount of data that HDFS can read or write is called a Block. The default block size is 64MB, but it can be increased as per the need to change in HDFS configuration.

Goals of HDFS
Fault detection and recovery : Since HDFS includes a large number of commodity hardware, failure of components is frequent. Therefore HDFS should have mechanisms for quick and automatic fault detection and recovery.

Huge datasets : HDFS should have hundreds of nodes per cluster to manage the applications having huge datasets.

Hardware at data : A requested task can be done efficiently, when the computation takes place near the data. Especially where huge datasets are involved, it reduces the network traffic and increases the throughput.

=


%-----------------------------------------------------%
\section{Hadoop Architecture Overview}
The core of Apache Hadoop consists of a storage part, known as Hadoop Distributed File System (HDFS), and a processing part called MapReduce. Hadoop splits files into large blocks and distributes them across nodes in a cluster.
Hadoop is supported by GNU/Linux platform and its flavors. Therefore, we have to install a Linux operating system for setting up Hadoop environment. In case you have an OS other than Linux, you can install a Virtualbox software in it and have Linux inside the Virtualbox.

Pre-installation Setup
Before installing Hadoop into the Linux environment, we need to set up Linux using ssh (Secure Shell). Follow the steps given below for setting up the Linux environment.

Creating a User
At the beginning, it is recommended to create a separate user for Hadoop to isolate Hadoop file system from Unix file system. Follow the steps given below to create a user:

Open the root using the command “su”.
Create a user from the root account using the command “useradd username”.
Now you can open an existing user account using the command “su username”.
Open the Linux terminal and type the following commands to create a user.

$ su 
   password: 
# useradd hadoop 
# passwd hadoop 
   New passwd: 
   Retype new passwd 
SSH Setup and Key Generation
SSH setup is required to do different operations on a cluster such as starting, stopping, distributed daemon shell operations. To authenticate different users of Hadoop, it is required to provide public/private key pair for a Hadoop user and share it with different users.

The following commands are used for generating a key value pair using SSH. Copy the public keys form id_rsa.pub to authorized_keys, and provide the owner with read and write permissions to authorized_keys file respectively.

$ ssh-keygen -t rsa 
$ cat ~/.ssh/id_rsa.pub >> ~/.ssh/authorized_keys 
$ chmod 0600 ~/.ssh/authorized_keys 
Installing Java
Java is the main prerequisite for Hadoop. First of all, you should verify the existence of java in your system using the command “java -version”. The syntax of java version command is given below.

$ java -version 
If everything is in order, it will give you the following output.

java version "1.7.0_71" 
Java(TM) SE Runtime Environment (build 1.7.0_71-b13) 
Java HotSpot(TM) Client VM (build 25.0-b02, mixed mode)  
If java is not installed in your system, then follow the steps given below for installing java.

Step 1
Download java (JDK <latest version> - X64.tar.gz) by visiting the following link http://www.oracle.com/technetwork/java/javase/downloads/jdk7-downloads1880260.html.

Then jdk-7u71-linux-x64.tar.gz will be downloaded into your system.

Step 2
Generally you will find the downloaded java file in Downloads folder. Verify it and extract the jdk-7u71-linux-x64.gz file using the following commands.

$ cd Downloads/ 
$ ls 
jdk-7u71-linux-x64.gz 
$ tar zxf jdk-7u71-linux-x64.gz 
$ ls 
jdk1.7.0_71   jdk-7u71-linux-x64.gz 
Step 3
To make java available to all the users, you have to move it to the location “/usr/local/”. Open root, and type the following commands.

$ su 
password: 
# mv jdk1.7.0_71 /usr/local/ 
# exit 
Step 4
For setting up PATH and JAVA_HOME variables, add the following commands to ~/.bashrc file.

export JAVA_HOME=/usr/local/jdk1.7.0_71 
export PATH=$PATH:$JAVA_HOME/bin 
Now apply all the changes into the current running system.

$ source ~/.bashrc
Step 5
Use the following commands to configure java alternatives:

# alternatives --install /usr/bin/java java usr/local/java/bin/java 2

# alternatives --install /usr/bin/javac javac usr/local/java/bin/javac 2

# alternatives --install /usr/bin/jar jar usr/local/java/bin/jar 2

# alternatives --set java usr/local/java/bin/java

# alternatives --set javac usr/local/java/bin/javac

# alternatives --set jar usr/local/java/bin/jar
Now verify the java -version command from the terminal as explained above.

Downloading Hadoop
Download and extract Hadoop 2.4.1 from Apache software foundation using the following commands.

$ su 
password: 
# cd /usr/local 
# wget http://apache.claz.org/hadoop/common/hadoop-2.4.1/ 
hadoop-2.4.1.tar.gz 
# tar xzf hadoop-2.4.1.tar.gz 
# mv hadoop-2.4.1/* to hadoop/ 
# exit 
Hadoop Operation Modes
Once you have downloaded Hadoop, you can operate your Hadoop cluster in one of the three supported modes:

Local/Standalone Mode : After downloading Hadoop in your system, by default, it is configured in a standalone mode and can be run as a single java process.

Pseudo Distributed Mode : It is a distributed simulation on single machine. Each Hadoop daemon such as hdfs, yarn, MapReduce etc., will run as a separate java process. This mode is useful for development.

Fully Distributed Mode : This mode is fully distributed with minimum two or more machines as a cluster. We will come across this mode in detail in the coming chapters.

Installing Hadoop in Standalone Mode
Here we will discuss the installation of Hadoop 2.4.1 in standalone mode.

There are no daemons running and everything runs in a single JVM. Standalone mode is suitable for running MapReduce programs during development, since it is easy to test and debug them.

Setting Up Hadoop
You can set Hadoop environment variables by appending the following commands to ~/.bashrc file.

export HADOOP_HOME=/usr/local/hadoop 
Before proceeding further, you need to make sure that Hadoop is working fine. Just issue the following command:

$ hadoop version 
If everything is fine with your setup, then you should see the following result:

Hadoop 2.4.1 
Subversion https://svn.apache.org/repos/asf/hadoop/common -r 1529768 
Compiled by hortonmu on 2013-10-07T06:28Z 
Compiled with protoc 2.5.0
From source with checksum 79e53ce7994d1628b240f09af91e1af4 
It means your Hadoop's standalone mode setup is working fine. By default, Hadoop is configured to run in a non-distributed mode on a single machine.

Example
Let's check a simple example of Hadoop. Hadoop installation delivers the following example MapReduce jar file, which provides basic functionality of MapReduce and can be used for calculating, like Pi value, word counts in a given list of files, etc.

$HADOOP_HOME/share/hadoop/mapreduce/hadoop-mapreduce-examples-2.2.0.jar 
Let's have an input directory where we will push a few files and our requirement is to count the total number of words in those files. To calculate the total number of words, we do not need to write our MapReduce, provided the .jar file contains the implementation for word count. You can try other examples using the same .jar file; just issue the following commands to check supported MapReduce functional programs by hadoop-mapreduce-examples-2.2.0.jar file.

$ hadoop jar $HADOOP_HOME/share/hadoop/mapreduce/hadoop-mapreduceexamples-2.2.0.jar 
Step 1
Create temporary content files in the input directory. You can create this input directory anywhere you would like to work.

$ mkdir input 
$ cp $HADOOP_HOME/*.txt input 
$ ls -l input 
It will give the following files in your input directory:

total 24 
-rw-r--r-- 1 root root 15164 Feb 21 10:14 LICENSE.txt 
-rw-r--r-- 1 root root   101 Feb 21 10:14 NOTICE.txt
-rw-r--r-- 1 root root  1366 Feb 21 10:14 README.txt 
These files have been copied from the Hadoop installation home directory. For your experiment, you can have different and large sets of files.

Step 2
Let's start the Hadoop process to count the total number of words in all the files available in the input directory, as follows:

$ hadoop jar $HADOOP_HOME/share/hadoop/mapreduce/hadoop-mapreduceexamples-2.2.0.jar  wordcount input output 
Step 3
Step-2 will do the required processing and save the output in output/part-r00000 file, which you can check by using:

$cat output/* 
It will list down all the words along with their total counts available in all the files available in the input directory.

"AS      4 
"Contribution" 1 
"Contributor" 1 
"Derivative 1
"Legal 1
"License"      1
"License");     1 
"Licensor"      1
"NOTICE”        1 
"Not      1 
"Object"        1 
"Source”        1 
"Work”    1 
"You"     1 
"Your")   1 
"[]"      1 
"control"       1 
"printed        1 
"submitted"     1 
(50%)     1 
(BIS),    1 
(C)       1 
(Don't)   1 
(ECCN)    1 
(INCLUDING      2 
(INCLUDING,     2 
.............
Installing Hadoop in Pseudo Distributed Mode
Follow the steps given below to install Hadoop 2.4.1 in pseudo distributed mode.

Step 1: Setting Up Hadoop
You can set Hadoop environment variables by appending the following commands to ~/.bashrc file.

export HADOOP_HOME=/usr/local/hadoop 
export HADOOP_MAPRED_HOME=$HADOOP_HOME 
export HADOOP_COMMON_HOME=$HADOOP_HOME 
export HADOOP_HDFS_HOME=$HADOOP_HOME 
export YARN_HOME=$HADOOP_HOME 
export HADOOP_COMMON_LIB_NATIVE_DIR=$HADOOP_HOME/lib/native 
export PATH=$PATH:$HADOOP_HOME/sbin:$HADOOP_HOME/bin 
export HADOOP_INSTALL=$HADOOP_HOME 
Now apply all the changes into the current running system.

$ source ~/.bashrc 
Step 2: Hadoop Configuration
You can find all the Hadoop configuration files in the location “$HADOOP_HOME/etc/hadoop”. It is required to make changes in those configuration files according to your Hadoop infrastructure.

$ cd $HADOOP_HOME/etc/hadoop
In order to develop Hadoop programs in java, you have to reset the java environment variables in hadoop-env.sh file by replacing JAVA_HOME value with the location of java in your system.

export JAVA_HOME=/usr/local/jdk1.7.0_71
The following are the list of files that you have to edit to configure Hadoop.

core-site.xml

The core-site.xml file contains information such as the port number used for Hadoop instance, memory allocated for the file system, memory limit for storing the data, and size of Read/Write buffers.

Open the core-site.xml and add the following properties in between <configuration>, </configuration> tags.

<configuration>

   <property>
      <name>fs.default.name </name>
      <value> hdfs://localhost:9000 </value> 
   </property>
 
</configuration>
hdfs-site.xml

The hdfs-site.xml file contains information such as the value of replication data, namenode path, and datanode paths of your local file systems. It means the place where you want to store the Hadoop infrastructure.

Let us assume the following data.

dfs.replication (data replication value) = 1 
(In the below given path /hadoop/ is the user name. 
hadoopinfra/hdfs/namenode is the directory created by hdfs file system.) 
namenode path = //home/hadoop/hadoopinfra/hdfs/namenode 
(hadoopinfra/hdfs/datanode is the directory created by hdfs file system.) 
datanode path = //home/hadoop/hadoopinfra/hdfs/datanode 
Open this file and add the following properties in between the <configuration> </configuration> tags in this file.

<configuration>

   <property>
      <name>dfs.replication</name>
      <value>1</value>
   </property>
    
   <property>
      <name>dfs.name.dir</name>
      <value>file:///home/hadoop/hadoopinfra/hdfs/namenode </value>
   </property>
    
   <property>
      <name>dfs.data.dir</name> 
      <value>file:///home/hadoop/hadoopinfra/hdfs/datanode </value> 
   </property>
       
</configuration>
Note: In the above file, all the property values are user-defined and you can make changes according to your Hadoop infrastructure.

yarn-site.xml

This file is used to configure yarn into Hadoop. Open the yarn-site.xml file and add the following properties in between the <configuration>, </configuration> tags in this file.

<configuration>
 
   <property>
      <name>yarn.nodemanager.aux-services</name>
      <value>mapreduce_shuffle</value> 
   </property>
  
</configuration>
mapred-site.xml

This file is used to specify which MapReduce framework we are using. By default, Hadoop contains a template of yarn-site.xml. First of all, it is required to copy the file from mapred-site,xml.template to mapred-site.xml file using the following command.

$ cp mapred-site.xml.template mapred-site.xml 
Open mapred-site.xml file and add the following properties in between the <configuration>, </configuration>tags in this file.

<configuration>
 
   <property> 
      <name>mapreduce.framework.name</name>
      <value>yarn</value>
   </property>
   
</configuration>
Verifying Hadoop Installation
The following steps are used to verify the Hadoop installation.

Step 1: Name Node Setup
Set up the namenode using the command “hdfs namenode -format” as follows.

$ cd ~ 
$ hdfs namenode -format 
The expected result is as follows.

10/24/14 21:30:55 INFO namenode.NameNode: STARTUP_MSG: 
/************************************************************ 
STARTUP_MSG: Starting NameNode 
STARTUP_MSG:   host = localhost/192.168.1.11 
STARTUP_MSG:   args = [-format] 
STARTUP_MSG:   version = 2.4.1 
...
...
10/24/14 21:30:56 INFO common.Storage: Storage directory 
/home/hadoop/hadoopinfra/hdfs/namenode has been successfully formatted. 
10/24/14 21:30:56 INFO namenode.NNStorageRetentionManager: Going to 
retain 1 images with txid >= 0 
10/24/14 21:30:56 INFO util.ExitUtil: Exiting with status 0 
10/24/14 21:30:56 INFO namenode.NameNode: SHUTDOWN_MSG: 
/************************************************************ 
SHUTDOWN_MSG: Shutting down NameNode at localhost/192.168.1.11 
************************************************************/
Step 2: Verifying Hadoop dfs
The following command is used to start dfs. Executing this command will start your Hadoop file system.

$ start-dfs.sh 
The expected output is as follows:

10/24/14 21:37:56 
Starting namenodes on [localhost] 
localhost: starting namenode, logging to /home/hadoop/hadoop
2.4.1/logs/hadoop-hadoop-namenode-localhost.out 
localhost: starting datanode, logging to /home/hadoop/hadoop
2.4.1/logs/hadoop-hadoop-datanode-localhost.out 
Starting secondary namenodes [0.0.0.0]
Step 3: Verifying Yarn Script
The following command is used to start the yarn script. Executing this command will start your yarn daemons.

$ start-yarn.sh 
The expected output as follows:

starting yarn daemons 
starting resourcemanager, logging to /home/hadoop/hadoop
2.4.1/logs/yarn-hadoop-resourcemanager-localhost.out 
localhost: starting nodemanager, logging to /home/hadoop/hadoop
2.4.1/logs/yarn-hadoop-nodemanager-localhost.out 
Step 4: Accessing Hadoop on Browser
The default port number to access Hadoop is 50070. Use the following url to get Hadoop services on browser.

http://localhost:50070/Accessing Hadoop on Browser
Step 5: Verify All Applications for Cluster
The default port number to access all applications of cluster is 8088. Use the following url to visit this service.

http://localhost:8088/



%-----------------------------------------------------%
\section{Terminology}
the cluster is the set of host machines (nodes). Nodes may be partitioned in racks. This is the hardware part of the infrastructure.
\section{Updated Version}
\section{NameNode and DataNode}

%-----------------------------------------------------%
\section{YARN for Apache Hadoop}

The YARN infrastructure and the HDFS federation are completely decoupled and independent: the first one provides resources for running an application while the second one provides storage. The MapReduce framework is only one of many possible framework which runs on top of YARN (although currently is the only one implemented).


%-----------------------------------------------------%
\section{HDFS}
Hadoop Distributed File System (HDFS) – a distributed file-system that stores data on commodity machines, providing very high aggregate bandwidth across the cluster;

The Hadoop distributed file system (HDFS) is a distributed, scalable, and portable file system written in Java for the Hadoop framework. Some consider HDFS to instead be a data store due to its lack of POSIX compliance and inability to be mounted,[64] but it does provide shell commands and Java API methods that are similar to other file systems.[65] A Hadoop cluster has nominally a single namenode plus a cluster of datanodes, although redundancy options are available for the namenode due to its criticality. Each datanode serves up blocks of data over the network using a block protocol specific to HDFS. The file system uses TCP/IP sockets for communication. Clients use remote procedure call (RPC) to communicate between each other.

HDFS stores large files (typically in the range of gigabytes to terabytes[66]) across multiple machines. It achieves reliability by replicating the data across multiple hosts, and hence theoretically does not require RAID storage on hosts (but to increase I/O performance some RAID configurations are still useful). With the default replication value, 3, data is stored on three nodes: two on the same rack, and one on a different rack. Data nodes can talk to each other to rebalance data, to move copies around, and to keep the replication of data high. HDFS is not fully POSIX-compliant, because the requirements for a POSIX file-system differ from the target goals for a Hadoop application. The trade-off of not having a fully POSIX-compliant file-system is increased performance for data throughput and support for non-POSIX operations such as Append.[67]

The HDFS file system is not restricted to MapReduce jobs. It can be used for other applications, many of which are under development at Apache. The list includes the HBase database, the Apache Mahout machine learning system, and the Apache Hive Data Warehouse system. 


%-----------------------------------------------------%
\subsection{Features of HDFS}
\begin{itemize}
\item It is suitable for the distributed storage and processing.
\item Hadoop provides a command interface to interact with HDFS.
\item The built-in servers of namenode and datanode help users to easily check the status of cluster.
Streaming access to file system data.
\item HDFS provides file permissions and authentication.
\end{itemize}
%-----------------------------------------------------%
\section{HDFS Federation}
The HDFS Federation is the framework responsible for providing permanent, 
reliable and distributed storage. This is typically used for storing inputs and output
 (but not intermediate ones).

%-----------------------------------------------------%
\section{JobTracker and TaskTracker: the MapReduce engine}

Above the file systems comes the MapReduce Engine, which consists of one JobTracker, to which client applications submit MapReduce jobs. The JobTracker pushes work out to available TaskTracker nodes in the cluster, striving to keep the work as close to the data as possible. With a rack-aware file system, the JobTracker knows which node contains the data, and which other machines are nearby. If the work cannot be hosted on the actual node where the data resides, priority is given to nodes in the same rack. This reduces network traffic on the main backbone network. If a TaskTracker fails or times out, that part of the job is rescheduled. The TaskTracker on each node spawns a separate Java Virtual Machine process to prevent the TaskTracker itself from failing if the running job crashes its JVM. A heartbeat is sent from the TaskTracker to the JobTracker every few minutes to check its status. The Job Tracker and TaskTracker status and information is exposed by Jetty and can be viewed from a web browser.

Hadoop streaming is a utility that comes with the Hadoop distribution. This utility allows you to create and run Map/Reduce jobs with any executable or script as the mapper and/or the reducer.

\subsection{Example Using Python}
For Hadoop streaming, we are considering the word-count problem. Any job in Hadoop must have two phases: mapper and reducer. We have written codes for the mapper and the reducer in python script to run it under Hadoop. One can also write the same in Perl and Ruby.

Mapper Phase Code
!/usr/bin/python
import sys
# Input takes from standard input for myline in sys.stdin: 
# Remove whitespace either side myline = myline.strip() 
# Break the line into words words = myline.split() 
# Iterate the words list for myword in words: 
# Write the results to standard output print '%s\t%s' % (myword, 1)
Make sure this file has execution permission (chmod +x /home/ expert/hadoop-1.2.1/mapper.py).

Reducer Phase Code
#!/usr/bin/python
from operator import itemgetter 
import sys 
current_word = ""
current_count = 0 
word = "" 
# Input takes from standard input for myline in sys.stdin: 
# Remove whitespace either side myline = myline.strip() 
# Split the input we got from mapper.py word, count = myline.split('\t', 1) 
# Convert count variable to integer 
   try: 
      count = int(count) 
except ValueError: 
   # Count was not a number, so silently ignore this line continue
if current_word == word: 
   current_count += count 
else: 
   if current_word: 
      # Write result to standard output print '%s\t%s' % (current_word, current_count) 
   current_count = count
   current_word = word
# Do not forget to output the last word if needed! 
if current_word == word: 
   print '%s\t%s' % (current_word, current_count)
Save the mapper and reducer codes in mapper.py and reducer.py in Hadoop home directory. Make sure these files have execution permission (chmod +x mapper.py and chmod +x reducer.py). As python is indentation sensitive so the same code can be download from the below link.

\subsection{Execution of WordCount Program}
$ $HADOOP_HOME/bin/hadoop jar contrib/streaming/hadoop-streaming-1.
2.1.jar \
   -input input_dirs \ 
   -output output_dir \ 
   -mapper <path/mapper.py \ 
   -reducer <path/reducer.py
Where "\" is used for line continuation for clear readability.

For Example,
./bin/hadoop jar contrib/streaming/hadoop-streaming-1.2.1.jar -input myinput -output myoutput -mapper /home/expert/hadoop-1.2.1/mapper.py -reducer /home/expert/hadoop-1.2.1/reducer.py
\subsection{How Streaming Works}
In the above example, both the mapper and the reducer are python scripts that read the input from standard input and emit the output to standard output. The utility will create a Map/Reduce job, submit the job to an appropriate cluster, and monitor the progress of the job until it completes.

When a script is specified for mappers, each mapper task will launch the script as a separate process when the mapper is initialized. As the mapper task runs, it converts its inputs into lines and feed the lines to the standard input (STDIN) of the process. In the meantime, the mapper collects the line-oriented outputs from the standard output (STDOUT) of the process and converts each line into a key/value pair, which is collected as the output of the mapper. By default, the prefix of a line up to the first tab character is the key and the rest of the line (excluding the tab character) will be the value. If there is no tab character in the line, then the entire line is considered as the key and the value is null. However, this can be customized, as per one need.

When a script is specified for reducers, each reducer task will launch the script as a separate process, then the reducer is initialized. As the reducer task runs, it converts its input key/values pairs into lines and feeds the lines to the standard input (STDIN) of the process. In the meantime, the reducer collects the line-oriented outputs from the standard output (STDOUT) of the process, converts each line into a key/value pair, which is collected as the output of the reducer. By default, the prefix of a line up to the first tab character is the key and the rest of the line (excluding the tab character) is the value. However, this can be customized as per specific requirements.

\subsection{Important Commands}
%----------------------------------------------%
\begin{center}
\begin{tabular}{|c|c|} \hline
	\textbf{Parameters	} &	\textbf{Description}	\\ \hline
\texttt{ -input directory/file-name	} &	Input location for mapper. (Required)	\\ \hline
\texttt{ -output directory-name	} &	Output location for reducer. (Required)	\\ \hline
\texttt{ -mapper executable or script or JavaClassName	} &	Mapper executable. (Required)	\\ \hline
\texttt{ -reducer executable or script or JavaClassName	} &	Reducer executable. (Required)	\\ \hline
\texttt{ -file file-name	} &	Makes the mapper, reducer, or combiner executable available locally on the compute nodes.	\\ \hline
\texttt{ -inputformat JavaClassName	} &	Class you supply should return key/value pairs of Text class. If not specified, TextInputFormat is used as the default.	\\ \hline
\texttt{ -outputformat JavaClassName	} &	Class you supply should take key/value pairs of Text class. If not specified, TextOutputformat is used as the default.	\\ \hline
\texttt{ -partitioner JavaClassName	} &	Class that determines which reduce a key is sent to.	\\ \hline
\texttt{ -combiner streamingCommand or JavaClassName	} &	Combiner executable for map output.	\\ \hline
\texttt{ -cmdenv name=value	} &	Passes the environment variable to streaming commands.	\\ \hline
\texttt{ -inputreader	} &	For backwards-compatibility: specifies a record reader class (instead of an input format class).	\\ \hline
\texttt{ -verbose	} &	Verbose output.	\\ \hline
\texttt{ -lazyOutput	} &	Creates output lazily. For example, if the output format is based on FileOutputFormat, the output file is created only on the first call to output.collect (or Context.write).	\\ \hline
\texttt{ -numReduceTasks	} &	Specifies the number of reducers.	\\ \hline
\texttt{ -mapdebug	} &	Script to call when map task fails.	\\ \hline
\texttt{ -reducedebug	} &	Script to call when reduce task fails.	\\ \hline
\end{tabular}
\end{center}

%-----------------------------------------------------%
\section{Hadoop High Availability}
